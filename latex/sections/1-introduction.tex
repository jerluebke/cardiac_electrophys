\section{Introduction}
The human heart is a fascinating apparatus, which does its work in a
constant and reliable fashion - usually without disruption - for the whole
of a persons life. To put things into perspective, the heart of an average
human being performs
\begin{center}
\begin{tabular}[t]{C R l}
            & 70    & (typical rest pulse per minute) \\
    \times  & 1440  & (minutes per day) \\
    \times  & 365   & (days per year) \\
    \times  & 80    & (estimated average human life time) \\
    \sim    & \mathbf{3\times10^9} & beats per life time,
\end{tabular}
\end{center}
while a typical car engine performs
\begin{center}
\begin{tabular}[t]{C R l}
            & \num{300000}  & (km drive during the cars life time) \\
    /       & 50    & (km/h, average speed) \\
    \times  & 60    & (minutes per hour) \\
    \times  & \num{2200}    & (revolutions per minute) \\
    \sim    & \mathbf{8\times10^8} & duty cycles per life time.
\end{tabular}
\end{center}

Investigating the hearts physical working principle poses an interesting
challenge with undoubtedly many relevant applications such as gaining a deeper
understanding of and developing more advanced treatments for often dangerous
arrhythmia.

The heart muscles basic functionality is rhythmically contracting itself
triggered by electrical signals, which are being conducted by the heart muscle
cells (the cardiomyocytes) themselves: the cardiac tissue combines the ability
to both perform mechanical work and conduct electrical signals.  This is very
remarkable, because usually these tasks are separately take care of by muscle
and neural cells, respectively.

Going a little more into detail: pacemaker cells (specialized cardiomyocytes)
in the SA-node rhythmically generate action potentials which travel at about
\SIrange{.05}{1}{\metre\per\second} to the AV-node and from there after a delay
at about \SIrange{2}{4}{\metre\per\second} through the His and ventricular
bundles and the Purkinje fibres, eventually reaching the intended tissue and
causing its contraction.


\subsection{Structure of the Cardiomyocytes}
For a better understanding it is helpful to take a closer look at the
microscopic structure of the cardiomyocytes:

Cardiomyocytes are tubular cells, which are enclosed by the \emph{sarcolemma}
(a double lipid-layered membrane) and contain chains of \emph{myofibril}
(fibres composed of long proteins), which are responsible for contraction of
the muscle tissue.  Additionally the cells contain the \emph{sarcoplastic
reticulum}, which is a membrane-enclosed region, mainly for storing
\ce{Ca^{2+}} ions.

In longitudinal direction \emph{interlacing disks} join the cells together
and via the \emph{gap junctions} allow propagation of action potentials.
Because of these features the heart muscle forms a \emph{syncytium}, \ie~the
whole of the single cells behaves like a single coordinated unit.


\subsection{Modeling the membrane potential}
Now the interior and exterior (\ie~the intermediate space between
neighbouring cells) regions of a cells exhibit different concentrations of
various ion species (this imbalance is being maintained by specialized ion
pumps and gates in the cell membrane), which results in a voltage between
those regions: the membrane potential $V=\Phi_i-\Phi_e$, which in the rest
case is equal to some rest potential $V_{\mathrm{rest}}$.

If at some point the membrane potential is perturbed by a stimulus in such
a way that it exceeds some threshold, the ion channels rapidly open causing
the concentration difference of the ions between interior and exterior
cell regions to invert resulting in a large upswing of the membrane
potential. This process is called \emph{depolarization} and the peak of the
membrane potential is called \emph{action potential}.
After reaching this peak the gates close again and the pumps recreate the
prior concentration difference which causes the membrane potential to
return to the rest value (\emph{repolarization}).

Since the membrane posses a finite specific electric capacity $C\,
[\si{\farad\per\metre\squared}]$,
the membrane potential obeys the capacitor equation:
\begin{equation}
    I=C\,\dv{V}{t}=-\sum_{s}I_{s}
    \label{eq:cap}
\end{equation}
were the total membrane current density $I$ is the sum of all involved
species-specific current densities.
Therewith the dynamics of $V$ can be modeled by modeling the
membrane current densities $I_s$.

\begin{figure}[h!]
    \centering
    \begin{circuitikz}
    \draw                               (5, 0)
        to[short,o-,l=intracellular]    (5, 1)
        to[short]                       (1, 1)
        to[C=$C_m$]                     (1, 5.5)
        to[short]                       (5, 5.5)
        to[short,-o,l=extracellular]    (5, 6.5);

    \draw                               (3, 5.5)
        to[short,i=$I_{\mathrm{Na}}$]   (3, 4)
        to[vR=Na gate]                  (3, 2.5)
        to[V,v=$V_{\mathrm{Na}}$]       (3, 1);

    \draw                               (6, 5.5)
        to[short,i=$I_{\mathrm{K}}$]    (6, 4)
        to[vR=K gate]                   (6, 2.5)
        to[V,v=$V_{\mathrm{K}}$]        (6, 1);

    \draw                               (5, 5.5)
        to[short]                       (9, 5.5)
        to[short,i=$I_{\mathrm{Ca}}$]   (9, 4)
        to[vR=Ca gate]                  (9, 2.5)
        to[V,v=$V_{\mathrm{Ca}}$]       (9, 1)
        to[short]                       (5, 1);

\end{circuitikz}


% vim: set ff=unix tw=79 sw=4 ts=4 et ic ai :

    \caption{Electrical circuit representing the membrane structure}
    \label{fig:circut}
\end{figure}


\subsection{Developing a Continuum Description}
\label{sec:contdescr}
At a microscopic level the propagation of action potential is a discrete
process (from cell to cell). However looking at tissue at sufficiently large
scales, it can be viewed as continuous (\textrightarrow~\emph{functional
syncytium}). It is important to note the anisotropic nature of this process:
the tissue exhibits different conductivities in longitudinal and transversal
direction with respect to the myofibril.

\subsubsection{Bidomain model}
One formulates potentials $\Phi_i, \Phi_e$ and current densities $\myvec{J_i},
\myvec{J_e}$ for the intra- and extracellular regions.  It is important to
note, that formally all of these functions are defined on the whole domain.

To set them into relation, consider Possion's equation and Ohm's Law:
\begin{gather*}
    \myvec{E}=-\nabla\Phi,\quad \myvec{J}=\myten{G}\,\myvec{E} \\
    \implies \myvec{J_i}=-\myten{G_i}\,\nabla\Phi_i,\quad
    \myvec{J_e}=-\myten{G_e}\,\nabla\Phi_e
\end{gather*}
where \myvec{E} is the electrical field associated with the potential $\Phi$
and \myten{G} is the conductivity tensor accounting for the anisotropy.

Imposing conservation of current:
\begin{equation*}
    \nabla\cdot(\myvec{J_i}+\myvec{J_e})=0 \implies
    -\nabla\cdot\myvec{J_i}=\nabla\cdot\myvec{J_e}=I_m
\end{equation*}
(where $I_m$ is the transmembrane current density with units
\si{\ampere\per\metre\cubed}), and rewriting the capacitor equation
\eqref{eq:cap}:
\begin{equation*}
    I_m=\beta\left(C\,\pdv{V}{t}+\sum_{s}I_{s}\right),\quad V=\Phi_i-\Phi_e
\end{equation*}
(where $\beta$ is a scaling constant with units \si{\per\metre}), one finds
after some calculations:
\begin{subequations}
\begin{align}
    \nabla\cdot\myten{G_i}(\nabla{V}+\nabla\Phi_e)
    &=\beta\left(C\,\pdv{V}{t}+I_{\mathrm{ion}}\right)
    \label{eq:i}
    \\
    \nabla\cdot\left((\myten{G_i}+\myten{G_e})\,\nabla\Phi_e\right)
    &=-\nabla\cdot(\myten{G_i}\,\nabla{V})
    \label{eq:ii}
\end{align}
\end{subequations}

Now one has a system of two coupled PDEs with \eqref{eq:i} being parabolic
and \eqref{eq:ii} being elliptic, which is rather difficult to solve.

\subsubsection{Monodomain model}
In order to make matters more accessible, one makes the assumption that the
intra- and extracellular anisotropies are identical, \ie~the respective
conductivities are proportional:
\begin{gather}
    \myten{G_e}=\lambda\,\myten{G_i} \nonumber \\ \implies
    \pdv{V}{t}=\nabla\cdot\myten{D}\,\nabla{V}-\frac{1}{C}\sum_{s}I_{s}
    \label{eq:divcap}
\end{gather}
(with the conductivity tensor $\myten{D}=\frac{1}{C\beta}\myten{G}$)
which reduces the problem to one parabolic PDE (a diffusion equation).

Now all that is left to do is to model the conductivity tensor \myten{D} to
represent the tissue at hand.
One way of doing this is to split this object
into components parallel and perpendicular to the direction of the
myofibril \myvec{f}, like so
\begin{equation*}
    \myten{D}=\myten{D_{\perp}}\,\mathbb{1}+(\myten{D_{\parallel}}
    -\myten{D_{\perp}})\,\myvec{f}\,\myvec{f^T}
\end{equation*}
where for typical cardiomyocytes one has the relation
$D_{\parallel}/D_{\perp}\sim2\ldots10$.

Another way is to neglect the anisotropies all together and write the
conductivity tensor as a scalar value: $\myten{D}\to\eta$. This is the
approach taken by the following investigations.

\vspace{\baselineskip}
\emph{Note}: Without external stimuli (enforced by von Neumann boundary
conditions) bi- and monodomain models yield almost identical results.
However when considering such external stimuli (\eg~defibrillation) the
unequal anisotropies of intra- and extracellular regions are significant.


% vim: set ff=unix tw=79 sw=4 ts=4 et ic ai :
