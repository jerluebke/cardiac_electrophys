\newcommand{\Vtilde}{\ensuremath{\tilde{V}}}

\section{Three Models}
In this section I am going to describe three approaches -- out of an
abundance of available models (\url{www.cellml.org}) -- which can be used to
describe the dynamics of the membrane potential.

\subsection{Hudgkin \& Huxley (1952)}
This model was developed by A.~L.~Hodgkin and A.~F.~Huxley to fit
measurements taken on a giant squid axon prior to detailed knowledge about
the biophysical mechanisms being available.

The membrane current density is modeled as the sum of Sodium, Potassium and
a leakage current, each obeying Ohm's Law:
\begin{gather}
    I_m = \sum_{s}I_{s}=I_{\mathrm{Na}}+I_{\mathrm{K}}+I_{\mathrm{leak}} \\
    I_s = g_s\,(V-V_s) \nonumber
\end{gather}
where the specific conductivites are described by dimensionless gating
variables $n, m, h\in[0,1]$:
\begin{equation*}
    g_{\mathrm{Na}}=\bar{g}_{\mathrm{Na}}\,n^4,\quad
    g_{\mathrm{K}}=\bar{g}_{\mathrm{K}}\,m^3\,h,\quad
    g_{\mathrm{leak}}=\bar{g}_{\mathrm{leak}}
\end{equation*}

The respective rest potentials and maximal specific conductivities were measured as:
\begin{table}[h!]
    \centering
    \begin{tabular}{c | C C}
        \toprule
        Current & V_s/\si{\milli\volt} &
        \bar{g}_s/\si{\milli\siemens\per\centi\metre\squared} \\
        \midrule
        Na      & 115   & 120   \\
        K       & -12   & 36    \\
        leak    & 10    & 0.3   \\
        \bottomrule
    \end{tabular}
\end{table}

The gating variables obey the following ODEs:
\begin{subequations}
\begin{align}
    \dv{n}{t}&=\alpha_{n}\,(1-n)-\beta_{n} \label{eq:n} \\
    \dv{m}{t}&=\alpha_{m}\,(1-m)-\beta_{m} \label{eq:m} \\
    \dv{h}{t}&=\alpha_{h}\,(1-h)-\beta_{h} \label{eq:h}
\end{align}
\end{subequations}
with ($\Vtilde=V/\si{\milli\volt}$):
\begin{align*}
    \alpha_{n}&=\SI{.01}{\per\milli\second}
        \frac{\Vtilde-10}{1-
            \exp\left(\frac{10-\Vtilde}{10}\right)},&
    \beta_{n}&=\SI{.125}{\per\milli\second}
        \exp\left(-\frac{\Vtilde}{80}\right) \\
    \alpha_{m}&=\SI{.1}{\per\milli\second}
        \frac{\Vtilde-25}{1-
            \exp\left(\frac{25-\Vtilde}{10}\right)},&
    \beta_{m}&=\SI{4}{\per\milli\second}
        \exp\left(-\frac{\Vtilde}{18}\right) \\
    \alpha_{h}&=\SI{.07}{\per\milli\second}
        \exp\left(-\frac{\Vtilde}{20}\right),&
    \beta_{h}&=\frac{1}{1+
        \exp\left(\frac{30-\Vtilde}{10}\right)}
\end{align*}

Thus, one has to solve a system of four first-order uncoupled ODEs
(\ref{eq:cap}, \ref{eq:n}, \ref{eq:m}, \ref{eq:h}).


\subsection{Aliev \& Panfilov (1996)}


\subsection{Fenton et al (2002)}


% vim: set ff=unix tw=79 sw=4 ts=4 et ic ai :
