\newcommand{\Vtilde}{\ensuremath{\tilde{V}}}

\section{Three Models}
In this section I am going to describe three approaches -- out of an
abundance of available models (\url{www.cellml.org}) -- which can be used to
describe the dynamics of the membrane potential.

\subsection{Hodgkin \& Huxley (1952)}
This model was developed by A.~L.~Hodgkin and A.~F.~Huxley to fit
measurements taken on a giant squid axon prior to detailed knowledge about
the biophysical mechanisms being available.

The membrane current density is modeled as the sum of Sodium, Potassium and
a leakage current, each obeying Ohm's Law:
\begin{gather}
    I_m = \sum_{s}I_{s}=I_{\mathrm{Na}}+I_{\mathrm{K}}+I_{\mathrm{leak}} \\
    I_s = g_s\,(V-V_s) \nonumber
\end{gather}
where the specific conductivites are described by dimensionless gating
variables $n, m, h\in[0,1]$:
\begin{equation*}
    g_{\mathrm{Na}}=\bar{g}_{\mathrm{Na}}\,n^4,\quad
    g_{\mathrm{K}}=\bar{g}_{\mathrm{K}}\,m^3\,h,\quad
    g_{\mathrm{leak}}=\bar{g}_{\mathrm{leak}}
\end{equation*}

The respective rest potentials and maximal specific conductivities were measured as:
\begin{table}[h!]
    \centering
    \begin{tabular}{c | C C}
        \toprule
        Current & V_s/\si{\milli\volt} &
        \bar{g}_s/\si{\milli\siemens\per\centi\metre\squared} \\
        \midrule
        Na      & 115   & 120   \\
        K       & -12   & 36    \\
        leak    & 10    & 0.3   \\
        \bottomrule
    \end{tabular}
\end{table}

The gating variables obey the following ODEs:
\begin{subequations}
\begin{align}
    \dv{n}{t}&=\alpha_{n}\,(1-n)-\beta_{n} \label{eq:n} \\
    \dv{m}{t}&=\alpha_{m}\,(1-m)-\beta_{m} \label{eq:m} \\
    \dv{h}{t}&=\alpha_{h}\,(1-h)-\beta_{h} \label{eq:h}
\end{align}
\end{subequations}
with ($\Vtilde=V/\si{\milli\volt}$):
\begin{align*}
    \alpha_{n}&=\SI{.01}{\per\milli\second}
        \frac{\Vtilde-10}{1-
            \exp\left(\frac{10-\Vtilde}{10}\right)},&
    \beta_{n}&=\SI{.125}{\per\milli\second}
        \exp\left(-\frac{\Vtilde}{80}\right) \\
    \alpha_{m}&=\SI{.1}{\per\milli\second}
        \frac{\Vtilde-25}{1-
            \exp\left(\frac{25-\Vtilde}{10}\right)},&
    \beta_{m}&=\SI{4}{\per\milli\second}
        \exp\left(-\frac{\Vtilde}{18}\right) \\
    \alpha_{h}&=\SI{.07}{\per\milli\second}
        \exp\left(-\frac{\Vtilde}{20}\right),&
    \beta_{h}&=\frac{1}{1+
        \exp\left(\frac{30-\Vtilde}{10}\right)}
\end{align*}

Thus, one has to solve a system of four first-order uncoupled ODEs
(\ref{eq:cap}, \ref{eq:n}, \ref{eq:m}, \ref{eq:h}).


\subsection{Aliev \& Panfilov (1996)}
While the Hodkin-Huxley model gives a very good description of
action potential dynamics, it is rather complex and therefor not preferable
for large scale computations.

An alternative model was formulated by Aliev and Panfilov, which refrains
from using a description based on biophysical details and instead uses two
variables (the potential $V$ and a relaxation variable $W$) to
phenomenologically reproduce the membrane potential dynamics of
cardiomyocytes:
\begin{subequations}
\begin{align}
    \dv{V}{t}&=-k\,V\,(V-a)\,(V-1)-V\,W \label{eq:ap_pot} \\
    \dv{W}{t}&=\epsilon(V, W)\,(-k\,V\,(V-a-1)-W) \label{eq:ap_relax} \\
    \epsilon&=\epsilon_0+\frac{\mu_1\,W}{V+\mu_2} \nonumber
\end{align}
\end{subequations}

The relaxation variable $W$ summarizes (or hides) all the complex processes
involving ion pumps etc., which cause the membrane potential to repolarize.

Here one has to solve two coupled first-order ODEs.

\subsection{Fenton et al (2002)}
Yet another model to be introduced here is again based on the capacitor
equation \eqref{eq:cap}. Unlike the models presented above, this approach
allows to easily investigate chaotic wave breakup mechanisms (its technical
meaning and possible physiological interpretation will be discussed in section
and section \todo{add section references} respectively).

The membrane current density is modeled as the sum of the following
phenomenological current densities:
\begin{itemize}
    \item fast inward current
        \[I_{fi}=-v\,\Theta(V-V_c)\,(V-V_c)\,(1-V)/\tau_d\]
    \begin{itemize}
        \item depolarizes membrane upon an excitation above $V_c$
        \item depends on a fast activation mechanism $\Theta(V-V_c)$ modeled by
            a Heaviside step function and the fast inactivation gate $v$
    \end{itemize}

    \item slow outward current
        \[I_{so}=V\,(1-\Theta(V-V_c))/\tau_0+\Theta(V-V_c)/\tau_r\]
    \begin{itemize}
        \item repolarizes membrane back to resting potential
        \item depends on fast activation mechanism $\Theta(V-V_c)$
    \end{itemize}

    \item slow inward current
        \[I_{si}=-w\,\frac{d}{2\,\tau_{si}},\quad
            d\to 1+\tanh(k\,(V-V_{c}^{si}))\]
    \begin{itemize}
        \item inactivation current to balance $I_{so}$
        \item produces observed plateau in action potential
        \item depends on slow inactivation gate $w$ and on a very fast
            activation gate d, which is modeled by a steady-state
            function
    \end{itemize}
\end{itemize}

The two gate variables governing the currents:
\begin{itemize}
    \item fast inactivation gate
        \begin{gather}
            \dv{v}{t}=(1-\Theta(V-V_c))\,(1-v)/\tau_{v-}
            -\Theta(V-V_c)\,v/\tau_{v+} \label{eq:v} \\
            \qq*{with}
            \tau_{v-}=(1-\Theta(V-V_v))\,\tau_{v-,1}
                +\Theta(V-V_v)\,\tau_{v-,2} \nonumber
        \end{gather}
    \item slow inactivation gate
        \begin{equation}
            \dv{w}{t}=(1-\Theta(V-V_c))\,(1-w)/\tau_{w-}
            -\Theta(V-V_c)\,w/\tau_{w+} \label{eq:w}
        \end{equation}
\end{itemize}

And finally the parameters:
\begin{itemize}
    \item $\tau_{v+}, \tau_{v-,1}, \tau_{v-,2}$: opening (+) and closing (-)
        times of the fast variable $v$
    \item $\tau_{w+}, \tau_{w-}$: opening and closing times of the slow
        variable $w$
    \item $\tau_d, \tau_r$: de- and repolarization times
    \item $\tau_0, \tau_{si}$: time constants for slow currents
    \item $V_c, V_v, V_{c}^{si}$: voltage thresholds
    \item $k$: activation width parameter
\end{itemize}

The problem to be solved here is composed of three uncoupled first-order
ODEs (\ref{eq:cap}, \ref{eq:v}, \ref{eq:w}).


% vim: set ff=unix tw=79 sw=4 ts=4 et ic ai :
