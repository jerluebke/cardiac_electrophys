\section{Physiological Interpretation}
\label{sec:phys}
\subsection{One-dimensional ring}
The one-dimensional ring described in section~\ref{sec:1d} can be interpreted
as a conducting fibre of the hearts conduction system, through which a constant
rate of action potentials passes. This rate can be varied by varying the length
of the ring. While this approach seems rather simplistic, it can be used to
understand the effects of tachycardia (\ie~abnormal high heart rates) on the
characteristics of single action potentials: the peaks are more shallow (longer
rise time), do not reach their full height, have a shorter duration and are
being conducted somewhat slower.

Such deviations from the regular behaviour can express itself as arrhythmia and
have a significant impact on the well-being of the affected person.


\subsection{Channel}
The situation depicted in section~\ref{sec:channel} of a confined channel could
be used to describe a bundle of cells belonging to the electrical conducting
system of the heart, which transmit action potentials at a constant rate to
some target tissue area, in order to trigger its contraction.

While with the default configuration this mimics a properly working propagation
of action potentials, one can also observe above described phenomena when
shortening the length of the domain.


\subsection{Spiral waves}
In a healthy heart the action potential meant to trigger contraction is
propagating homogeneously through the tissue. If now due to some pathological
condition some regions in the cardiac tissue have different electrical
conductivities or do not conduct at all (scar tissue), the action potential
might travel around these obstacles eventually hitting its own wake and causing
local depolarization deviating from the beat given by the SA node.  Such
phenomena are known as \emph{re-entrant arrhythmia} and can cause ventricular
tachycardia or atrial flutter and might develop into ventricular or atrial
fibrillation.  The only way to stop (the otherwise fatal) ventricular
fibrillation is to apply a high energy electrically shock to depolarize all (or
a critical mass) of cardiac tissue (\textrightarrow~\emph{defibrillation}).
When the cells have repolarized, they can again be governed by the pace given
by the SA node.

Such phenomena can be linked to action potentials spreading through the heart's
tissue like spiral waves, which in reality occur due to heterogeneity of the
tissue. On a simplified homogeneous domain, one can generate such
configurations by choosing the initial conditions properly. The setups
presented above achieve this by delivering a single impulse in the wake of a
preceding action potential wave (see section~\ref{sec:spiral1}) or by defining
an initial wave with a loose end (see section~\ref{sec:spiral2}). The spiral
waves resulting from the first case and occurring in the beginning of
the second case can be interpreted as ventricular tachycardia; the disruptive
and uncoordinated behaviour eventually emerging from the second case resembles
ventricular fibrillation evolving from preceding tachycardia (see also fig.~38
in \cite{Cherry2008}).


\section{Discussion}
\subsection{The Models}
\subsubsection{Hodgkin \& Huxley}
The first-discussed model is very intuitive and gives a good understanding of
the underlying process, since it is close to the physical background (even
though the details were unknown at that time). Even hyperpolarization is
included.

The results presented here were obtained with a time step of
$\Delta{t}=\SI{.01}{\milli\second}$. Reasonable results were possible with
larger time steps up to $\Delta{t}=\SI{.055}{\milli\second}$, however at this
point the results already included artifacts due to numerical inaccuracies.

While the form of the action potential rather corresponds to neural cells and
not cardiomyocytes, this could perhaps be achieved by fitting the parameters
differently. The additional source current causing repeated depolarization
can be used to describe permanently firing neurons or -- with
appropriate parameters -- pacemaker cells in the SA node.

\subsubsection{Aliev \& Panfilov}
While this model is rather difficult to interpret, since it is intended to
only phenomenologically reproduce the observed dynamics, it allows one to
perform larger spatial tissue simulations with reduced effort.

From this models simulation the measurements of the action potential properties
for varying rate were recorded, which were compared to a theoretical estimate
relating conduction velocity and rise time \eqref{eq:cvrt}.
While the gross form of the theoretical estimate and the measured data is
similar, a rather clear deviation should be acknowledged. Quantitatively
-- from the compared slopes -- a relative error of 48\% is found. In order to
make a clear statement whether the error is due to inaccuracies of the equation
or of the used model, a more in-depth investigation would be needed.

\subsubsection{Fenton et al}
This model was introduced, because the previously discussed A\&P model lacks
the ability to depict more complex (and interesting) mechanisms. This comes
with the price of a higher computational demand.

It is per se a phenomenological model, yet the used quantities (\emph{currents}
and \emph{gates}) resemble the underlying physics closely. The parameters have
physical meaning and are designed in such a way, that allows to influence the
resulting behaviour purposefully.

\subsection{Further Outlook}
Throughout this work all simulations were performed with a simple Euler scheme.
While this already gives meaningful results, one could alternatively apply more
sophisticated ways such as a semi-implicit Crank-Nicolson scheme or
multi-step Runge-Kutta methods. This would allow for larger time steps (making
the simulations computationally cheaper) while maintaining accuracy.

It might be interesting to consider a heterogeneous domain, \ie~with
anisotropic conductivity. One could investigate the behaviour of above
discussed setups with a conductivity tensor resembling the structure of the
myofibril. Such a simulation could be expanded to more complex geometries in
order to describe fibre bundles of the conduction system or re-entrant waves
due to scar tissue.

Further, developing solution methods for the bi-domain case would allow the
investigation of the tissue's behaviour when subjected to external stimuli such
as defibrillation.


% vim: set ff=unix tw=79 sw=4 ts=4 et ic ai :
