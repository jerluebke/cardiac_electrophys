\section{Physiological Interpretation}
\label{sec:phys}
\subsection{One-dimensional ring}
The one-dimensional ring described in section~\ref{sec:1d} can be interpreted
as a conducting fibre of the hearts conduction system, through which a constant
rate of action potentials passes. This rate can be varied by varying the length
of the ring. While this approach seems rather simplistic, it can be used to
understand the effects of tachycardia (\ie~abnormal high heart rates) on the
characteristics of the single action potentials: the peak is more shallow
(longer rise time), does not reach its full height, has a shorter duration and
is being conducted somewhat slower.

Such deviations from the regular behaviour can express itself as arrhythmia and
have a significant impact on the well-being of affected person.


\subsection{Channel}
The situation depicted in section~\ref{sec:channel} of a confined channel could
be used to describe a bundle of cells belonging to the electrical conducting
system of the heart, which transmits action potentials at a constant rate to
some target tissue area, in order to trigger its contraction.

While with the default configuration this mimics a properly working propagation
of action potentials, one can also observe above described phenomena when
shortening the length of the domain.


\subsection{Spiral waves}
In a healthy heart the action potential meant to trigger contraction is
propagating homogeneously through tissue. If now due to some pathological
condition some regions in the cardiac tissue have different electrical
conductivities or do not conduct at all (scar tissue), the action potential
might travel around these obstacles eventually hitting its own wake and causing
local depolarization deviating from the beat given by the SA node:
\emph{re-entrant arrhythmia}. Such phenomena can cause ventricular tachycardia
or atrial flutter and might develop into ventricular or atrial fibrillation.
The only way to stop (the otherwise fatal) ventricular fibrillation is to
apply a high energy electrically shock to depolarize all (or a
critical mass) of cardiac tissue (defibrillation). When the cells have
repolarized, they can again be governed by the pacemaker cells of the SA node.

Those aforementioned phenomena resemble spiral waves, which in reality occur
due to heterogeneity of the tissue. On a simplified homogeneous domain, one can
generate such spiral waves \eg~by delivering a single impulse in the wake of a
propagating action potential wave (see section~\ref{sec:spiral1}) or by
defining an initial wave with a loose end (see section~\ref{sec:spiral2}).
While the spiral waves discussed in section~\ref{sec:spiral1} can be
interpreted as ventricular tachycardia, the disruptive and uncoordinated
dynamics seen in section~\ref{sec:spiral2} resemble ventricular fibrillation
(see also fig.~38 in \cite{Cherry2008}).


\section{Discussion}


% vim: set ff=unix tw=79 sw=4 ts=4 et ic ai :
